\section{PROJECT INTRODUCTION AND CONTEXT}

Lake Waikare, situated in the lower Waikato catchment of New Zealand, represents far more than a geographical feature—it embodies centuries of M\=aori cultural heritage, traditional ecological knowledge, and environmental stewardship that has sustained local communities for generations. However, contemporary environmental challenges threaten not only the lake's ecological integrity but also the cultural continuity of the knowledge systems intrinsically connected to this vital natural resource.

\subsection{Cultural and Environmental Crisis}

The environmental degradation of Lake Waikare due to agricultural runoff and inadequate wastewater management has created a cascading effect that extends beyond water quality metrics into the realm of cultural preservation. As the largest shallow lake in the lower Waikato catchment, Lake Waikare has historically served as a cornerstone for M\=aori communities, providing essential resources for sustenance and maintaining profound spiritual and cultural significance \cite{regional_council_2024}. The deterioration of this ecosystem has resulted in the gradual erosion of traditional knowledge, oral histories, and customary practices that have been transmitted through generations.

This phenomenon represents a critical intersection of environmental and cultural loss, where ecological degradation directly threatens the preservation of indigenous knowledge systems. The disconnection between younger generations and their cultural heritage, exacerbated by modernization processes, compounds this challenge by creating knowledge gaps that traditional transmission methods struggle to bridge effectively.

\subsection{Digital Preservation as Cultural Revitalization}

In response to these interconnected challenges, this project investigates the potential of digital technologies to serve as culturally appropriate preservation and revitalization tools. Unlike conventional digital archiving approaches that often treat cultural content as static artifacts, our research explores how interactive digital platforms can maintain the dynamic, relational, and contextual nature of M\=aori knowledge systems while ensuring accessibility for diverse community stakeholders.

The Lake Waikare Digital Library project leverages the Greenstone Digital Library Software as its foundational platform, recognizing Greenstone's established capabilities in cultural heritage preservation and its extensive deployment in indigenous communities worldwide \cite{greenstone_2024}. Developed by the New Zealand Digital Library Project at the University of Waikato, Greenstone provides a robust, open-source framework specifically designed for building and distributing digital library collections, making it particularly suitable for community-controlled cultural preservation initiatives.

The project emerges from extensive community consultation and represents a collaborative effort between academic researchers, local iwi, and community stakeholders to develop a preservation platform that respects traditional knowledge protocols while leveraging Greenstone's proven digital capabilities. This approach recognizes that effective cultural preservation requires more than mere documentation—it demands the creation of engaging, accessible, and culturally authentic platforms that can facilitate intergenerational knowledge transmission while maintaining the technical reliability and scalability that Greenstone provides.

\subsection{Research Significance and Innovation}

This project contributes to the growing field of digital heritage preservation by addressing several critical gaps in existing approaches while demonstrating innovative applications of the Greenstone Digital Library framework. By building upon Greenstone's established architecture, the project explores how existing digital library technologies can be enhanced and customized to serve indigenous knowledge preservation requirements more effectively.

First, the project demonstrates how Greenstone's flexible collection management system can be adapted to respect and reflect indigenous epistemologies rather than imposing Western knowledge organization systems. The platform's customizable interface capabilities enable the creation of culturally appropriate navigation structures and content presentation methods that align with M\=aori knowledge organization principles.

Second, it explores innovative approaches to intergenerational engagement through the development of specialized Child Mode interfaces within the Greenstone framework, demonstrating how established digital library platforms can be extended to make cultural content accessible to younger audiences without compromising authenticity or cultural protocols. This represents a novel application of Greenstone's interface customization capabilities for age-specific cultural engagement.

The integration of environmental and cultural preservation within a single Greenstone-based platform represents an innovative approach that recognizes the inseparable connection between ecological health and cultural vitality in M\=aori worldviews. By leveraging Greenstone's multimedia handling capabilities and metadata management systems, the platform presents environmental data alongside cultural narratives, demonstrating how digital library technologies can support holistic preservation approaches that inform contemporary environmental stewardship while maintaining cultural relevance for modern communities.

\subsection{Community-Centered Methodology}

Central to this project's approach is the recognition that authentic cultural preservation cannot occur without genuine community partnership and ongoing stakeholder engagement. The project operates under the guidance of Glen Tupuhi, our principal stakeholder and cultural authority, whose extensive governance experience and deep cultural connections provide essential direction for the platform's development.

Glen Tupuhi brings unparalleled expertise to this initiative through his role as Trustee of Whakatupu Aotearoa Foundation and his extensive background in M\=aori governance structures. His whakapapa connections—Tainui te waka, Ng\=aati P\=aoa ki Waiheke, T\=amaki Makaurau, Hauraki, Waikato, Ngati Hine, Ngati Naho o Waikato, Ngati Rangimahora, Ng\=aati Apakura—establish direct cultural authority over the Lake Waikare region and ensure that the digital library development respects appropriate cultural protocols and community priorities.

Glen's governance portfolio, including his roles as Hauraki representative for Waikato District Health Board Iwi M\=aori Council, Chair of Ng\=aa Muka Development Trust, and his previous positions with The Ngati P\=aoa Trust and Hauraki M\=aori Trust Board, demonstrates the collaborative networks essential for sustainable cultural preservation initiatives. His academic credentials, including a Graduate Diploma in Business Studies from Massey University and NZ Institute of Directors Certificate, provide the strategic oversight necessary for developing community-controlled digital resources.

Rather than adopting extractive research methodologies that treat communities as data sources, this project employs a collaborative framework where Glen Tupuhi and associated community members serve as co-researchers, cultural authorities, and primary beneficiaries of the developed platform. This methodology ensures that the digital library reflects community priorities, respects cultural protocols, and addresses genuine community needs rather than academic assumptions about cultural preservation requirements.

The involvement of Glen's extensive network, including connections to the Waikato Regional Council through his various trustee roles, demonstrates the project's commitment to creating sustainable, community-controlled resources that can continue to evolve and expand beyond the initial development phase.

\subsection{Project Scope and Objectives}

The Lake Waikare Digital Library project encompasses the design, development, and evaluation of a comprehensive digital platform built upon the Greenstone Digital Library Software framework. This foundation provides the technical infrastructure necessary to serve multiple community stakeholder groups while maintaining cultural authenticity, accessibility, and long-term sustainability.

The platform leverages Greenstone's core capabilities—including multimedia collection management, flexible metadata schemas, and customizable user interfaces—while extending these features through innovative adaptations for cultural preservation and community engagement. The project integrates interactive mapping technologies with Greenstone's search and browsing capabilities, implements multilingual content presentation using Greenstone's internationalization features, and develops sophisticated filtering systems that respect cultural organization principles.

The development of the specialized Child Mode represents a significant extension of Greenstone's interface capabilities, demonstrating how established digital library platforms can be enhanced to create age-appropriate cultural engagement tools while maintaining the robust collection management and preservation standards for which Greenstone is recognized.

This report presents the complete development process, from initial community consultation through final prototype evaluation, demonstrating how the Greenstone platform can be adapted and extended to address community-identified challenges while maintaining rigorous technical standards. The project's outcomes extend beyond the specific Lake Waikare context to provide insights and methodologies applicable to similar Greenstone-based digital heritage preservation initiatives in indigenous communities worldwide, contributing to the broader ecosystem of culturally responsive digital library implementations.